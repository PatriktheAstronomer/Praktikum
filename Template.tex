\documentclass[a4paper]{article}
\usepackage[utf8]{inputenc}
\usepackage[czech]{babel}
\usepackage[T1]{fontenc}
\usepackage{amsmath}
\usepackage{graphicx}
\usepackage{txfonts}
\usepackage{eurosym}
\usepackage[symbol*]{footmisc}
\usepackage{mathtools}
\usepackage{enumitem}
\usepackage{tabularx,ragged2e,booktabs,caption}
\author{"Patrik Novotný"}

\begin{document}
\section*{Pracovní úkol}
\begin{enumerate}[label=(\alph*)]
\item Nejprve jsme si přečetli studijí text...
\item Abychom později mohli...
\item Bla bla bla
\item ...a nakonec jsme dostali 20/20.
\end{enumerate}

% Poznámky se píšou s pomocí procenta.

\section*{Teorie}
\par Odvodili jsme následující vztah, kde t značí čas, A albedo, P výkon, respektive solární konstantu.
\begin{align}
E =(1-A)*P*3600*(t-t_{rise}) \int_{0}^{\delta_{time}} \sin(x)*dx
\end{align}
\section*{Výsledky měření}
\begin{center}
    \captionof{table}{Předpověď počasí} \label{tab:title} 
    \begin{tabular}{ | l | l | l | p{5cm} |}
    \hline
    Day [0] & Min Temp [C] & Max Temp [C] & Summary \\ \hline
    Monday & 11C & 22C & A clear day with lots of sunshine.  
    However, the strong breeze will bring down the temperatures. \\ \hline
    Tuesday & 9C & 19C & Cloudy with rain, across many northern regions. Clear spells 
    across most of Scotland and Northern Ireland, 
    but rain reaching the far northwest. \\ \hline
    Wednesday & 10C & 21C & Rain will still linger for the morning. 
    Conditions will improve by early afternoon and continue 
    throughout the evening. \\
    \hline
    \end{tabular}
\end{center}

\section*{Diskuse}
\section*{Závěr}
\par A nakonec nám došlo, že máme rádi MFF, protože má naprosto cool logo.
\begin{figure}[!ht]
\centering
\caption{LOGO MFF}
\includegraphics[width=230pt]{matfyz_barevne.png}
\end{figure}

\renewcommand\refname{Použitá literatura}
\begin{thebibliography}{}
\bibitem{latexcompanion} 
Michel Goossens, Frank Mittelbach, and Alexander Samarin. 
\textit{The \LaTeX\ Companion}. 
Addison-Wesley, Reading, Massachusetts, 1993.
 
\bibitem{einstein} 
Albert Einstein. 
\textit{Zur Elektrodynamik bewegter K{\"o}rper}. (German) 
[\textit{On the electrodynamics of moving bodies}]. 
Annalen der Physik, 322(10):891–921, 1905.
 
\bibitem{knuthwebsite} 
Knuth: Computers and Typesetting,
\\\texttt{http://www-cs-faculty.stanford.edu/\~{}uno/abcde.html}
\end{thebibliography}
\end{document}
